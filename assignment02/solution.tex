\documentclass[10pt,a4paper]{scrartcl}
\usepackage{../uebungsblatt}
\usepackage{dsfont}
\usepackage[utf8]{inputenc}
\usepackage{ngerman}
\usepackage{enumitem}
\usepackage{stmaryrd}
\usepackage{a4wide}
\usepackage[ruled, vlined]{algorithm2e}
\usepackage{amsmath}
\usepackage{dsfont}
\usepackage{tikz}
\usepackage{tikz-qtree}
\usepackage{float}
\usepackage{xcolor}
\usepackage{listings}

\lstset{
  moredelim=[is][\underbar]{-}{-}
}

\newcommand{\Pot}{\mathcal{P}}

\newcommand{\N}{\mathbb{N}}
\newcommand{\Z}{\mathbb{Z}}


%Kopfzeile:
\fancyhead[R]{Seite: \thepage \hspace{0.5ex} von \pageref{LastPage}}
\fancyhead[L]{Wildner \& Metzner }

\begin{document}

\uebkopfzeile
  {Digital Libraries} % Titel der Veranstaltung
  {WS 14/15}  % Semesterangabe, Übungsgruppe
  {}    % Dozenten, Übungsleiter
  {Jens Metzner \& Manuel Wildner}    % Loesungsblattbearbeiter

\uebtitel
{Solution of assignment 2} % Titel (gross und zentriert)
{21. Nov} % Datum der Abgabe


\solution{1}{Explain the 5 major components of an Ontology and discuss based on these components, the differences between an Ontology and a Classification}{5}

\solution{2}{How does the keyword based approach for retrieval in text documents work? What are its advantages and disadvantages?}{3}

\solution{3}{Define the measures of Precision and Recall by which we can assess the quality of a retrieval algorithm.}{5}

\solution{4}{Give two examples how precision and recall can be negatively affected in a keyword- based query.}{2}

\solution{5}{How can an Ontology be used to improve retrieval in text documents?}{2}

\solution{6}{Consider the tf*idf vector model for indexing text. Explain how the term weights are computed in this model, and give an interpretation of these values.}{3}

\end{document}
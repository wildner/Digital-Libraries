\documentclass[10pt,a4paper]{scrartcl}
\usepackage{../uebungsblatt}
\usepackage{dsfont}
\usepackage[utf8]{inputenc}
\usepackage{ngerman}
\usepackage{enumitem}
\usepackage{stmaryrd}
\usepackage{a4wide}
\usepackage[ruled, vlined]{algorithm2e}
\usepackage{amsmath}
\usepackage{dsfont}
\usepackage{tikz}
\usepackage{tikz-qtree}
\usepackage{float}
\usepackage{xcolor}
\usepackage{listings}

\lstset{
  moredelim=[is][\underbar]{-}{-}
}

\newcommand{\Pot}{\mathcal{P}}

\newcommand{\N}{\mathbb{N}}
\newcommand{\Z}{\mathbb{Z}}


%Kopfzeile:
\fancyhead[R]{Seite: \thepage \hspace{0.5ex} von \pageref{LastPage}}
\fancyhead[L]{Wildner \& Metzner }

\begin{document}

\uebkopfzeile
  {Digital Libraries} % Titel der Veranstaltung
  {WS 14/15}  % Semesterangabe, Übungsgruppe
  {}    % Dozenten, Übungsleiter
  {Jens Metzner \& Manuel Wildner}    % Loesungsblattbearbeiter

\uebtitel
{Solution of assignment 2} % Titel (gross und zentriert)
{21. Nov} % Datum der Abgabe


\solution{1}{Explain the 5 major components of an Ontology and discuss based on these components, the differences between an Ontology and a Classification}{5}
\begin{itemize}
	\item \textbf{Concepts:} A concept describes a single object or a set of objects like humans, animals or lifeless things. The words ''human'' and ''animal'' are a concept names.
	\item \textbf{Instances:} Instances are not descriptions but concrete objects. So the ''Konzil'' in Konstanz is an instance of the concept ''building''.
	\item \textbf{Properties:} Each object can have multiple attributes, called properties, like size, color, etc.
	\item \textbf{Relations:} This component describes the relation between two instances or sets of instances respectively. For example: The University of Konstanz is located in Konstanz.
	\item \textbf{Rules:} By rules it is possible to infer from one information (e.g. a property) that other properties are valid too. If someone drives a car, then he should have a driver license.
\end{itemize}
Classification is used to structure concepts hierarchically, for example by a tree-structure. A classification cannot represent other relations at the same time. So a classification contains less information, but can help to structure parts of the ontology.

\solution{2}{How does the keyword based approach for retrieval in text documents work? What are its advantages and disadvantages?}{3}

\solution{3}{Define the measures of Precision and Recall by which we can assess the quality of a retrieval algorithm.}{5}

\solution{4}{Give two examples how precision and recall can be negatively affected in a keyword- based query.}{2}

\solution{5}{How can an Ontology be used to improve retrieval in text documents?}{2}

\solution{6}{Consider the tf*idf vector model for indexing text. Explain how the term weights are computed in this model, and give an interpretation of these values.}{3}

\end{document}
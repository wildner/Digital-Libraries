\documentclass[10pt,a4paper]{scrartcl}
\usepackage{../uebungsblatt}
\usepackage{dsfont}
\usepackage[utf8]{inputenc}
\usepackage{ngerman}
\usepackage{enumitem}
\usepackage{stmaryrd}
\usepackage{a4wide}
\usepackage[ruled, vlined]{algorithm2e}
\usepackage{amsmath}
\usepackage{dsfont}
\usepackage{tikz}
\usepackage{tikz-qtree}
\usepackage{float}
\usepackage{xcolor}
\usepackage{listings}

\lstset{
  moredelim=[is][\underbar]{-}{-}
}

\newcommand{\Pot}{\mathcal{P}}

\newcommand{\N}{\mathbb{N}}
\newcommand{\Z}{\mathbb{Z}}


%Kopfzeile:
\fancyhead[R]{Seite: \thepage \hspace{0.5ex} von \pageref{LastPage}}
\fancyhead[L]{Wildner \& Metzner }

\begin{document}

\uebkopfzeile
  {Digital Libraries} % Titel der Veranstaltung
  {WS 14/15}  % Semesterangabe, Übungsgruppe
  {}    % Dozenten, Übungsleiter
  {Jens Metzner \& Manuel Wildner}    % Loesungsblattbearbeiter

\uebtitel
{Solution of assignment 2} % Titel (gross und zentriert)
{21. Nov} % Datum der Abgabe


\solution{1}{Explain the 5 major components of an Ontology and discuss based on these components, the differences between an Ontology and a Classification}{5}
\begin{itemize}
	\item \textbf{Concepts:} A concept describes a single object or a set of objects like humans, animals or lifeless things. The words ''human'' and ''animal'' are a concept names.
	\item \textbf{Instances:} Instances are not descriptions but concrete objects. So the ''Konzil'' in Konstanz is an instance of the concept ''building''.
	\item \textbf{Properties:} Each object can have multiple attributes, called properties, like size, color, etc.
	\item \textbf{Relations:} This component describes the relation between two instances or sets of instances respectively. For example: The University of Konstanz is located in Konstanz.
	\item \textbf{Rules:} By rules it is possible to infer from one information (e.g. a property) that other properties are valid too. If someone drives a car, then he should have a driver license.
\end{itemize}
Classification is used to structure concepts hierarchically, for example by a tree-structure. A classification cannot represent other relations at the same time. So a classification contains less information, but can help to structure parts of the ontology.

\solution{2}{How does the keyword based approach for retrieval in text documents work? What are its advantages and disadvantages?}{3}

\solution{3}{Define the measures of Precision and Recall by which we can assess the quality of a retrieval algorithm.}{5}
Precision and recall refer to the sets of relevant/non-relevant documents and the set of retrieved documents which consists of parts of the other two sets.\\
Precision compares the true positives (relevant retrieved) to all retrieved documents. Recall measures how many true positives were received compared to the whole relevant set.\\
\[
	precision = \frac{\# relevant\ retrieved}{\# retrieved} \qquad recall = \frac{\# relevant\ retrieved}{\# relevant\ in\ collection}
\]

\solution{4}{Give two examples how precision and recall can be negatively affected in a keyword- based query.}{2}

\solution{5}{How can an Ontology be used to improve retrieval in text documents?}{2}

\solution{6}{Consider the tf*idf vector model for indexing text. Explain how the term weights are computed in this model, and give an interpretation of these values.}{3}
\textbf{Computation:}
\begin{itemize}
	\item TF (term frequency): \# occurences of a term in the document
	\item DF (document frequency): \# occurences of a document throughout all documents
	\item TF*IDF: (term frequency)*(inverse document frequency)
\end{itemize}
\textbf{Interpretation:}
\begin{itemize}
	\item TF: The more times a term $t$ occurs in document $d$, the more likely $t$ is relevant to the document.
	\item DF: The more a term $t$ occurs throughout all documents, the more poorly $t$ discriminates between documents.
	\item TF*IDF: A high value indicates that the word occurs more often in this document than average.
\end{itemize}

\end{document}